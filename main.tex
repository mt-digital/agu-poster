%%%%%%%%%%%%%%%%%%%%%%%%%%%%%%%%%%%%%%%%%
% Jacobs Landscape Poster
% LaTeX Template
% Version 1.1 (14/06/14)
%
% Created by:
% Computational Physics and Biophysics Group, Jacobs University
% https://teamwork.jacobs-university.de:8443/confluence/display/CoPandBiG/LaTeX+Poster
%
% Further modified by:
% Nathaniel Johnston (nathaniel@njohnston.ca)
%
% This template has been downloaded from:
% http://www.LaTeXTemplates.com
%
% License:
% CC BY-NC-SA 3.0 (http://creativecommons.org/licenses/by-nc-sa/3.0/)
%
%%%%%%%%%%%%%%%%%%%%%%%%%%%%%%%%%%%%%%%%%

%----------------------------------------------------------------------------------------
%	PACKAGES AND OTHER DOCUMENT CONFIGURATIONS
%----------------------------------------------------------------------------------------

\documentclass[final]{beamer}

\usepackage[scale=1.24]{beamerposter} % Use the beamerposter package for laying out the poster

\usetheme{confposter} % Use the confposter theme supplied with this template

\setbeamercolor{block title}{fg=ngreen,bg=white} % Colors of the block titles
\setbeamercolor{block body}{fg=black,bg=white} % Colors of the body of blocks
\setbeamercolor{block alerted title}{fg=white,bg=dblue!70} % Colors of the highlighted block titles
\setbeamercolor{block alerted body}{fg=black,bg=dblue!10} % Colors of the body of highlighted blocks
% Many more colors are available for use in beamerthemeconfposter.sty

%-----------------------------------------------------------
% Define the column widths and overall poster size
% To set effective sepwid, onecolwid and twocolwid values, first choose how many columns you want and how much separation you want between columns
% In this template, the separation width chosen is 0.024 of the paper width and a 4-column layout
% onecolwid should therefore be (1-(# of columns+1)*sepwid)/# of columns e.g. (1-(4+1)*0.024)/4 = 0.22
% Set twocolwid to be (2*onecolwid)+sepwid = 0.464
% Set threecolwid to be (3*onecolwid)+2*sepwid = 0.708

\newlength{\sepwid}
\newlength{\onecolwid}
\newlength{\twocolwid}
\newlength{\threecolwid}
\setlength{\paperwidth}{48in} % A0 width: 46.8in
\setlength{\paperheight}{36in} % A0 height: 33.1in
\setlength{\sepwid}{0.024\paperwidth} % Separation width (white space) between columns
\setlength{\onecolwid}{0.22\paperwidth} % Width of one column
\setlength{\twocolwid}{0.464\paperwidth} % Width of two columns
\setlength{\threecolwid}{0.708\paperwidth} % Width of three columns
\setlength{\topmargin}{-0.5in} % Reduce the top margin size
%-----------------------------------------------------------

\usepackage{graphicx}  % Required for including images

\usepackage{booktabs} % Top and bottom rules for tables

%----------------------------------------------------------------------------------------
%	TITLE SECTION
%----------------------------------------------------------------------------------------

\title{A best-practices model for cyberinfrastructure collaboration, as applied to create the Coupled RipCAS-DFLOW model for floodplain modeling} % Poster title

\author{Matthew A. Turner$^{a,b}$, Sarah Miller$^{c,d}$, Angela Gregory$^d$,
        Smriti Chaulagain$^d$, Lucas J. Sheneman$^a$,
        Mark Stone$^d$, Daniel Cadol$^c$
} % Author(s)

\institute{$^a$Northwest Knowledge Network, University of Idaho, USA;
           $^b$Cognitive and Information Sciences, University of California, Merced, USA \\
           $^c$Department of Earth and Environmental Sciences, New Mexico Tech, USA;
           $^d$Environmental Laboratory, U.S. Army Engineer Research and Development Center, Vicksburg \\
           $^e$Department of Civil Engineering, University of New Mexico, USA
           } % Institution(s)

%----------------------------------------------------------------------------------------

\begin{document}

\addtobeamertemplate{block end}{}{\vspace*{2ex}} % White space under blocks
\addtobeamertemplate{block alerted end}{}{\vspace*{2ex}} % White space under highlighted (alert) blocks

\setlength{\belowcaptionskip}{2ex} % White space under figures
\setlength\belowdisplayshortskip{2ex} % White space under equations

\begin{frame}[t] % The whole poster is enclosed in one beamer frame

\begin{columns}[t] % The whole poster consists of three major columns, the second of which is split into two columns twice - the [t] option aligns each column's content to the top

\begin{column}{\sepwid}\end{column} % Empty spacer column

\begin{column}{\onecolwid} % The first column

%----------------------------------------------------------------------------------------
%	OBJECTIVES
%----------------------------------------------------------------------------------------

\begin{alertblock}{Project Goal}
Build a sustainable, easy-to-use, transparent modeling system for riparian zone flood modeling over
many years of successive flood events.
\end{alertblock}
\begin{alertblock}{Poster Objectives}

\begin{itemize}
    \item Introduce basic principles of effective software development, customized for small-team scientific/cyberinfrastructure collaboration
    \item Demonstrate how these principles resulted in an open-source coupled-model system that includes
        automatic data management: Coupled RipCAS-DFLOW (CoRD)
\end{itemize}

\end{alertblock}

%----------------------------------------------------------------------------------------
%	INTRODUCTION
%----------------------------------------------------------------------------------------

\begin{block}{Minimal Solution}
    \begin{itemize}
        \item Write adapters so data from each model is compatible with the other
        \item Create a way to run the coupled modeling system
        \item Write documentation on how to use the adapters
    \end{itemize}
\end{block}

\begin{block}{Short list of Best Practices (BP)}
    By following these best practices, software can be more valuable and do more than just
    couple two models.
    \begin{enumerate}
        \item{Work closely with scientists as development partners; commit to weekly meetings}
        \item{Deliver scientifically valuable software quickly in small chunks}
        \item{Do not increase the amount of programming or technical skill required for scientists to do their jobs.}
        \item{Openness is key: open-source software, built-in data management and data sharing.}
        \item{Use existing software wisely.}
        \item{Take an adaptive approach; avoid waterfall planning.}
    \end{enumerate}
    The rest of this poster is devoted to describing the deeper problem that
    next-generation modeling systems should be solving and how we are solving that deeper problem.
\end{block}


%\begin{block}{Deeper Problem}
    %Create a coupled model that is easy to use. Scientists should be free to focus on science, not on programming or learning
    %data management principles, techniques, and standards. In order for the new model to be externally verified, it must be
    %easy to use, the code must be effectively open sourced, and data sharing must be included. Too often the models that generate
    %published results require opaque, unpublished parameterizations. It’s not clear whether the compelling results are real or just an
    %artifact of the model.
%\end{block}

%------------------------------------------------

%----------------------------------------------------------------------------------------

\end{column} % End of the first column

\begin{column}{\sepwid}\end{column} % Empty spacer column

\begin{column}{\twocolwid} % Begin a column which is two columns wide (column 2)

%----------------------------------------------------------------------------------------
%	IMPORTANT RESULT
%----------------------------------------------------------------------------------------

\begin{alertblock}{Coupled RipCAS-DFLOW (CoRD) In Action}
    CoRD couples two models. The first is the DFLOW model for streamflow force modeling. The other is Riparian Community Alteration and
    Succession model, which we've developed as part of this process. In fact, following BP1), it was written entirely by the
    team at New Mexico Tech. \\
    Here is the timeline of progress of CoRD, which shows how we used BP2)
    \begin{itemize}
        \item Phase I: Developed I/O Adapters for single-year coupled modeling
        \item Phase II: Process for cycling through many years of flood intensity
        \item Phase III: Added command-line interface (CLI) for running many-years simulation
        \item Phase IV: Added automatic data management through HydroShare\cite{Horsburgh2016} REST API (see BP5)
        \item Phase V: Added automatic data processing and visualization, also written by hydrologists at NM Tech
        \item Phase VI: Web app interface for even simpler modeling on the CPU cluster and data management
    \end{itemize}


    \begin{figure}
        \includegraphics[width=0.4\linewidth]{placeholder.jpg}
        \caption{I'll have some combo of pictures of command-line interface, hydroshare upload, web app, and visualization routines}
    \end{figure}
\end{alertblock}

%----------------------------------------------------------------------------------------

\begin{columns}[t,totalwidth=\twocolwid] % Split up the two columns wide column

\begin{column}{\onecolwid}\vspace{-.6in} % The first column within column 2 (column 2.1)

%----------------------------------------------------------------------------------------
%	MATERIALS
%----------------------------------------------------------------------------------------


\begin{block}{Deeper Problem}
    \begin{itemize}
        \item Scientists are trained in analysis, but not in data management and open software practices
        \item Without open modeling software, it's difficult for models to be externally verified
        \item Modeling results often emerge from highly opaque parameter tunings, limiting the opportunities
            for other scientists to verify results
        \item Too often cyberinfrastructure collaborations are ``one-way''---developers don't know or appreciate
            scientists goals and skills
        \item Too often cyberinfrastructure collaborations lack iterative feedback between scientists and
            software developers
        \item Tools built without scientist participation frequently languish
    \end{itemize}

\end{block}

%----------------------------------------------------------------------------------------

\end{column} % End of column 2.1

\begin{column}{\onecolwid}\vspace{-.6in} % The second column within column 2 (column 2.2)

%----------------------------------------------------------------------------------------
%	METHODS
%----------------------------------------------------------------------------------------


%----------------------------------------------------------------------------------------

\end{column} % End of column 2.2

\end{columns} % End of the split of column 2 - any content after this will now take up 2 columns width

\end{column} % End of the second column

\begin{column}{\sepwid}\end{column} % Empty spacer column

\begin{column}{\onecolwid} % The third column

%----------------------------------------------------------------------------------------
%	CONCLUSION
%----------------------------------------------------------------------------------------

\begin{block}{Discussion}
    Monolithic projects like EarthCube and CSDMS are laudable efforts. But EarthCube ``is facing a mid-life crisis,''
    according to a recent report in Science. After five years, an external advisory panel ``warned that EarthCube
    still lacked a clear definition and might not be sustainable.''\cite{Witze2016} CSDMS requires high up-front investment without a
    clear payoff. Modern software development theory would predict that such ``waterfall'' approaches to software are
    doomed to sub-optimality \cite{Sutherland2014}. Instead, modern approaches lead us to adopt the steps I outlined above.
\end{block}

%----------------------------------------------------------------------------------------
%	ADDITIONAL INFORMATION
%----------------------------------------------------------------------------------------

%\begin{block}{Additional Information}
%
%Maecenas ultricies feugiat velit non mattis. Fusce tempus arcu id ligula varius dictum.
%\begin{itemize}
%\item Curabitur pellentesque dignissim
%\item Eu facilisis est tempus quis
%\item Duis porta consequat lorem
%\end{itemize}
%
%\end{block}

%----------------------------------------------------------------------------------------
%	REFERENCES
%----------------------------------------------------------------------------------------

\begin{block}{References}

%\nocite{*} % Insert publications even if they are not cited in the poster
\small{\bibliographystyle{unsrt}
\bibliography{bib}\vspace{0.75in}}

\end{block}

%----------------------------------------------------------------------------------------
%	ACKNOWLEDGEMENTS
%----------------------------------------------------------------------------------------

%\setbeamercolor{block title}{fg=red,bg=white} % Change the block title color

%\begin{block}{Acknowledgements}

%\small{\rmfamily{Nam mollis tristique neque eu luctus. Suspendisse rutrum congue nisi sed convallis. Aenean id neque dolor. Pellentesque habitant morbi tristique senectus et netus et malesuada fames ac turpis egestas.}} \\

%\end{block}

%----------------------------------------------------------------------------------------
%	CONTACT INFORMATION
%----------------------------------------------------------------------------------------

%\setbeamercolor{block alerted title}{fg=black,bg=norange} % Change the alert block title colors
%\setbeamercolor{block alerted body}{fg=black,bg=white} % Change the alert block body colors
%
%\begin{alertblock}{Contact Information}
%
%\begin{itemize}
%\item Web: \href{http://www.university.edu/smithlab}{http://www.university.edu/smithlab}
%\item Email: \href{mailto:john@smith.com}{john@smith.com}
%\item Phone: +1 (000) 111 1111
%\end{itemize}
%
%\end{alertblock}
%
%\begin{center}
%\begin{tabular}{ccc}
%\includegraphics[width=0.4\linewidth]{logo.png} & \hfill & \includegraphics[width=0.4\linewidth]{logo.png}
%\end{tabular}
%\end{center}
%
%%----------------------------------------------------------------------------------------
%
\end{column} % End of the third column
%
\end{columns} % End of all the columns in the poster

\end{frame} % End of the enclosing frame

\end{document}
